\chapter{Metodología}
\label{chap:metodologia}

Este capítulo describe el enfoque metodológico aplicado al diseño e implementación de la arquitectura conversacional multiagente para la gestión de comprobantes tributarios deducibles en el Perú. Se detalla el hardware y software empleado, la arquitectura técnica y de agentes, el flujo de API, los principios de diseño, el modelo de datos, los criterios de calidad y seguridad, así como el plan de evaluación y métricas.

\section{Enfoque general}
El sistema opera sobre dos rutas de procesamiento complementarias:
\begin{enumerate}
  \item \textbf{Ruta de archivos (pipeline de ingesta):} flujo determinista e idempotente para recibir PDFs/imágenes, ejecutar OCR y parsing, validar RUC/condición, clasificar deducibilidad y persistir resultados. Implementado como \emph{workflow} de cinco agentes en Agno.
  \item \textbf{Ruta de consultas (ReAct):} un agente conversacional planifica y ejecuta herramientas ORM para responder preguntas agregadas del usuario (filtros por RUC, fechas, montos, categoría, totales).
\end{enumerate}
La selección de ruta se basa en el tipo de solicitud (subida de archivos o mensaje de texto) y se ejecuta en un backend FastAPI que expone los endpoints \texttt{/comprobantes/subir} y \texttt{/comprobantes/consultar}.

\section{Hardware utilizado}
\begin{itemize}
  \item \textbf{Procesador:} CPU Intel Core i7-12700KF 16 nucleos
  \item \textbf{Memoria RAM:} 32 GB
  \item \textbf{Tarjeta gráfica:} GPU RTX 4070 super 12 GB
  \item \textbf{Dispositivo móvil de prueba:} Teléfono Android con 8 GB RAM

\section{Software utilizado}
\begin{itemize}
  \item \textbf{Backend:} Python 3.11, FastAPI, Agno (agentes y workflows), SQLAlchemy ORM, PostgreSQL 16.
  \item \textbf{OCR y PDF:} Pipeline 1 (modelo OCR local servido vía Ollama); Pipeline 2 (PaddleOCR para imágenes + PyMuPDF para PDFs nativos).
  \item \textbf{LLMs:} modelos locales en Ollama de distintos tamaños (4B, 8B, 12B, 20B) para parsing y razonamiento; uso comparativo de tamaños para balancear latencia/calidad.
  \item \textbf{Móvil:} Expo/React Native, \texttt{react-navigation}, \texttt{react-native-paper}; pantallas de login/registro, chat, carga de comprobantes y detalle de recibo.
  \item \textbf{Herramientas auxiliares:} librerías de hashing (\texttt{sha256}) para deduplicación, validación de RUC/condición vía servicio SUNAT, almacenamiento temporal de blobs y manejo de sesiones de usuario.
\end{itemize}

\section{Arquitectura}
\subsection{Agentes secuenciales (pipeline de ingesta)}
El \emph{workflow} \texttt{IngestaWorkflow} coordina cinco agentes:
\begin{enumerate}
  \item \textbf{Agente Validador de Comprobante:} calcula \texttt{sha256}, detecta duplicados por hash y por combinación (usuario, emisor, serie, número).
  \item \textbf{Agente Parseador:} ejecuta OCR según MIME o tipo de documento (\texttt{PyMuPDF} para PDF nativo; \texttt{PaddleOCR}/modelo local para imagen), invoca LLM local para estructurar campos y aplica \emph{fallback} heurístico con expresiones regulares.
  \item \textbf{Agente Validador SUNAT:} consulta estado y condición del RUC, obtiene CIIU y aplica reglas básicas de deducibilidad.
  \item \textbf{Agente Clasificador:} asigna categoría de gasto y porcentaje deducible según reglas SUNAT, registrando la versión de regla.
  \item \textbf{Agente Persistencia:} usa un \emph{prompt} con \emph{few shots} para revisar el contexto (emisor, comprobante, detalle, validación, clasificación, hash y flags) y detectar inconsistencias suaves (totales que no cuadran, serie/número vacíos, RUC ausente, fechas fuera de rango). La decisión de persistir sigue siendo determinista: si las validaciones duras pasan, se ejecuta el \emph{upsert} idempotente mediante repositorios SQLAlchemy y se confirma con \texttt{commit}; de lo contrario, se aborta con \texttt{rollback}.
\end{enumerate}


\subsection{Agente ReAct para consultas}
Un agente Agno con \texttt{tool\_choice} forzado utiliza \emph{QueryToolkit} para:
\begin{itemize}
  \item Búsqueda de comprobantes con filtros (RUC, fechas, categoría, montos, límite de resultados).
  \item Cálculo de totales y agregaciones (promedio, suma, conteo; agrupación opcional por categoría).
  \item Búsqueda por emisor (RUC o nombre) y detalle de comprobantes asociados.
\end{itemize}
Las herramientas operan sobre el ORM, retornan JSON estructurado y el agente genera una respuesta auditada con los resultados.

\begin{figure}[h!]
  \centering
  \includegraphics[width=\textwidth]{Figures/Arquitectura.png}
  \caption{Arquitectura general del sistema.}
  \label{fig:arquitectura}
\end{figure}


\section{Flujo de la API}
\begin{enumerate}
  \item \textbf{Subida de comprobantes:} el cliente móvil envía archivos a \texttt{/comprobantes/subir}, se ejecuta el pipeline de cinco pasos y se retorna: ID, hash, bandera de duplicado, campos parseados, validación SUNAT y clasificación.
  \item \textbf{Consultas:} el cliente envía texto a \texttt{/comprobantes/consultar}; el agente ReAct traduce la intención a filtros ORM, ejecuta las herramientas y retorna respuesta en lenguaje natural más datos estructurados.
\end{enumerate}

\begin{figure}[h!]
  \centering
  \includegraphics[width=\textwidth]{Figures/api.png}
  \caption{Flujo de la API.}
  \label{fig:flujoapi}
\end{figure}

\section{Patrones y principios de diseño}
\begin{itemize}
  \item \textbf{Arquitectura por funcionalidades (features):} agrupación por contexto de dominio, como autenticación y agentes.
  \item \textbf{Inversión de dependencias:} Los repositorios abstraen el ORM, mientras que los agentes consumiran las interfaces de de estos repositorios, evitando dependencia directa con el ORM.
  \item \textbf{Workflows en Agno:} Los workflows permiten crear flujos de agentes deterministas y controlados. Cada agente maneja una tarea especifica pesada, las salidas de cada agente son pasadas como argumentos al siguiente.
\end{itemize}

\section{Diseño de la base de datos y persistencia}
El esquema relacional incluye:
\begin{itemize}
  \item \texttt{usuario} (cuentas), \texttt{emisor} (RUC, razón social, CIIU, estado/condición).
  \item \texttt{comprobante} (tipo, serie, número, fecha, monto, moneda, origen, hash, estado, deducible, duplicado; FKs a usuario y emisor; unicidad por (usuario, emisor, serie, número) y por hash).
  \item \texttt{detalle\_comprobante} (ítems y montos), \texttt{validacion} (estado/condición RUC, CIIU, coincidencia de nombre, reglas), \texttt{clasificacion} (categoría, porcentaje, versión de regla).
  \item \texttt{ocr\_pagina} (texto por página y confianza), \texttt{estado\_trabajo} (seguimiento de trabajos), \texttt{historial\_chat} (mensajes de conversación).
\end{itemize}

\begin{figure}[h!]
  \centering
  \includegraphics[width=\textwidth]{Figures/DB.png}
  \caption{Esquema relacional de la base de datos.}
  \label{fig:bd}
\end{figure}

\section{Calidad, seguridad y cumplimiento}
\begin{itemize}
  \item \textbf{Privacidad:} uso de modelos locales en Ollama para evitar exponer boletas a servicios externos.
  \item \textbf{Seguridad:} autenticación con middleware JWT, cifrado en tránsito y manejo de sesiones por usuario.
  \item \textbf{Confiabilidad:} se valida la duplicidad de los comprobantes antes de procesarlos y se ejecuta un rollback en caso de fallos.
\end{itemize}

\section{Plan de evaluación y métricas}
\begin{itemize}
  \item \textbf{OCR:} exactitud por campo (RUC, razón social, serie, número, fecha, monto e items encontrados), CER/WER y confianza promedio; robustez ante distintas calidades de imagen.
  \item \textbf{Pipeline:} latencia extremo a extremo y \emph{throughput}; tasa de deduplicación efectiva.
  \item \textbf{LLMs:} tiempo de respuesta y consumo de GPU/RAM para tamaños 8B/12B/20B en tareas de parsing y consulta; calidad percibida de la respuesta conversacional.
  \item \textbf{Datos de prueba:} mezcla de comprobantes reales y sintéticos etiquetados; partición para evaluación; scripts de generación controlada para estrés de formatos.
\end{itemize}

\section{Riesgos y mitigaciones}
\begin{itemize}
  \item \textbf{Variabilidad de imágenes:} preprocesamiento (normalización, reducción de ruido, realce de contraste) y retroalimentación al usuario cuando la confianza OCR es baja.
  \item \textbf{Picos de carga:} colas y lotes internos para el procesamiento paralelo de archivos PDFs e imágenes.
\end{itemize}
