\chapter{Conclusiones y trabajo futuro}
\label{cap:conclusiones}

\section{Conclusiones generales}

El trabajo desarrollado demuestra la viabilidad de una arquitectura conversacional multiagente para asistir a personas naturales en la gestión de sus gastos deducibles en el contexto peruano. La combinación de extracción híbrida de texto (PyMuPDF para PDFs nativos y PP-OCRv5 para imágenes), validación básica con datos de SUNAT y un agente de consulta basado en el patrón ReAct permite ofrecer a la persona usuaria una visión más clara y trazable de sus comprobantes y de su nivel de deducibilidad acumulada.

Los experimentos realizados muestran que es posible alcanzar una alta precisión en la extracción de campos clave de los comprobantes y en la interpretación de la información oficial del RUC, manteniendo tiempos de respuesta aceptables para una aplicación interactiva. Asimismo, el agente de consulta evidencia que la interacción en lenguaje natural sobre datos estructurados es práctica y útil para consultas frecuentes sobre gastos y deducciones.

\section{Conclusiones por objetivo específico}

A continuación, se presentan las conclusiones asociadas a cada uno de los objetivos específicos planteados:

\begin{enumerate}
  \item \textbf{Diseñar e implementar una arquitectura conversacional multiagente basada en agentes secuenciales y en el patrón ReAct.}\\
  Se logró diseñar e implementar un conjunto de agentes especializados que coordinan la ingesta, validación y consulta de comprobantes. El flujo secuencial del pipeline de ingesta simplificó la trazabilidad y el manejo de errores, mientras que el patrón ReAct en los agentes de validación SUNAT y consulta permitió integrar razonamiento y uso de herramientas de manera flexible.

  \item \textbf{Diseñar e implementar un modelo de datos relacional en PostgreSQL, desacoplado mediante repositorios y un ORM.}\\
  Se definió un modelo relacional que captura las entidades clave del dominio (usuario, emisor, comprobante, detalle, validación, clasificación, OCR y trabajos en curso) y se implementó el patrón de repositorios para desacoplar la lógica de negocio del ORM. Este enfoque facilitó la prueba unitaria de los agentes y dejó abierta la posibilidad de cambiar la tecnología de persistencia en el futuro.

  \item \textbf{Construir conjuntos de datos de evaluación, combinando comprobantes sintéticos y comprobantes con RUC reales.}\\
  Se construyó un conjunto de 20 comprobantes sintéticos para la evaluación del parseador y un conjunto de 5 casos de prueba con RUC reales para el validador SUNAT, además de 4 consultas de prueba para el agente de consulta. Aunque el tamaño de los conjuntos es limitado, resultó suficiente para evidenciar diferencias claras en desempeño entre modelos y para validar el funcionamiento de la arquitectura.

  \item \textbf{Desarrollar y evaluar un agente de consulta en lenguaje natural basado en ReAct y herramientas SQL deterministas.}\\
  Se implementó un agente de consulta que, a partir de una pregunta en lenguaje natural, selecciona y parametriza herramientas SQL predefinidas sobre la base de datos. Los resultados muestran que el modelo \texttt{qwen3:4b} alcanzó una precisión del 100\% en la selección de herramientas y argumentos en los casos de prueba, demostrando la eficacia del enfoque ReAct en este contexto, aunque con una latencia superior a la de otros modelos.

  \item \textbf{Analizar comparativamente el desempeño de distintos modelos de lenguaje integrados en los agentes.}\\
  El análisis cuantitativo reveló que \texttt{llama3.1:8b} ofrece un excelente equilibrio entre latencia y precisión en tareas de generación estructurada, mientras que \texttt{qwen3:4b} destacó en tareas de razonamiento y uso de herramientas, especialmente en la extracción de códigos CIIU y en la interpretación de consultas complejas. \texttt{gemma3:12b} mostró un comportamiento intermedio, con buena precisión pero ciertas limitaciones en su compatibilidad con herramientas en el entorno de ejecución.

  \item \textbf{Implementar una aplicación móvil con Expo/React Native que consuma los servicios del backend multiagente.}\\
  Se desarrolló una aplicación móvil basada en Expo que expone pantallas de autenticación, subida de comprobantes y chat con el agente de consulta. La elección de Expo permitió acelerar el ciclo de desarrollo y facilita la futura incorporación de actualizaciones OTA. Aunque la evaluación de usabilidad queda fuera del alcance cuantitativo de esta tesis, la integración técnica entre la aplicación y el backend multiagente se completó satisfactoriamente.
\end{enumerate}

\section{Limitaciones}

Entre las principales limitaciones del trabajo se encuentran:

\begin{itemize}
  \item El tamaño reducido de los conjuntos de datos de evaluación, que si bien permiten observar tendencias claras, no garantizan representatividad sobre la amplia variedad de formatos de comprobantes y situaciones tributarias reales.
  \item La ausencia de una comparación directa con una arquitectura base sin orquestación multiagente (por ejemplo, un único modelo de lenguaje monolítico), lo que impide cuantificar de forma precisa la ganancia atribuible a la arquitectura agentic frente a alternativas más simples.
  \item El uso de reglas simplificadas para la deducibilidad, centradas en el estado y condición del RUC y en un mapeo aproximado por CIIU, que no cubren todas las particularidades de la normativa peruana ni reemplazan el juicio profesional de un contador.
  \item La falta de evaluación formal de la experiencia de usuario en la aplicación móvil, incluyendo aspectos de usabilidad, comprensión de las respuestas del agente y percepción de confiabilidad.
\end{itemize}

\section{Trabajo futuro}

A partir de los resultados y limitaciones identificados, se plantean varias líneas de trabajo futuro:

\begin{itemize}
  \item \textbf{Ampliación del conjunto de evaluación}: recolectar un corpus más amplio y diverso de comprobantes reales (respetando normas de protección de datos) que cubra distintos giros de negocio, formatos y calidades de imagen, así como escenarios de error frecuentes.
  \item \textbf{Comparación con arquitecturas alternativas}: implementar y evaluar una arquitectura de referencia sin orquestación multiagente (por ejemplo, un único modelo de lenguaje con prompts complejos) para cuantificar de manera rigurosa los beneficios en precisión, trazabilidad y mantenibilidad de la solución agentic.
  \item \textbf{Enriquecimiento de reglas tributarias}: incorporar reglas más detalladas de deducibilidad y tope de gastos, incluyendo el seguimiento dinámico de la deducción adicional de 3 UIT, y evaluar el impacto de estas reglas en la utilidad práctica del sistema.
  \item \textbf{Agente de persistencia inteligente}: dotar al agente de persistencia de capacidades conversacionales para detectar campos vacíos o inconsistentes, proponer valores sugeridos mediante el modelo de lenguaje y solicitar confirmación al usuario antes de registrar cambios.
  \item \textbf{Extensión de funcionalidades financieras}: ampliar el sistema para que, además de gestionar gastos deducibles, permita a cualquier persona gestionar sus gastos, ingresos y ahorros de forma más general, utilizando el mismo pipeline de ingesta y la interfaz conversacional.
  \item \textbf{Integración con flujos de declaración}: explorar la generación asistida de borradores de declaración jurada anual, en coordinación con la normativa de SUNAT, manteniendo siempre un enfoque de apoyo al usuario y no de automatización plena del proceso de declaración.
\end{itemize}

En conjunto, estas líneas de trabajo apuntan a evolucionar la arquitectura presentada desde un prototipo funcional orientado a la gestión de comprobantes hacia una plataforma integral de asistencia financiera y tributaria basada en agentes conversacionales.
