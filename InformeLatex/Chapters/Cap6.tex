\chapter{Conclusiones y trabajo futuro}
\label{cap:conclusiones}

Diseñamos un sistema conversacional multiagente para automatizar la captura, validación y clasificación de comprobantes deducibles en el contexto peruano. La solución integra lectura de PDFs nativos con PyMuPDF y OCR para escaneos con PaddleOCR, aplica reglas tributarias versionadas y valida contra fuentes de SUNAT, todo orquestado bajo principios de Arquitectura Limpia/DDD que separan claramente dominio, aplicación e infraestructura. El modelo de datos preserva evidencia (texto OCR, archivos, huellas) y asegura idempotencia mediante claves naturales y SHA-256, habilitando trazabilidad y auditoría.

Metodológicamente, seguiremos un enfoque incremental-iterativo: (1) definición de requerimientos y criterios de éxito (exactitud OCR, concordancia de clasificación, latencia, cobertura de validaciones); (2) diseño de la arquitectura agentic con un pipeline determinístico de ingesta y una ruta de consultas; (3) implementación del MVP con SQLAlchemy/SQLite y camino de migración a PostgreSQL; (4) pruebas unitarias y de integración centradas en idempotencia, deduplicación y validación contra casos reales; y (5) evaluación con métricas objetivas y registro exhaustivo para análisis posterior. Este proceso permitira ajustar decisiones tecnológicas con evidencia (por ejemplo, selección OCR por tipo de documento) y establecer una base sólida para el escalamiento y el endurecimiento futuro del sistema.
