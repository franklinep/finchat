\chapter{Estado del Arte}

El desarrollo reciente de arquitecturas basadas en \emph{agentes inteligentes} ha transformado la interacción de los modelos de lenguaje con entornos complejos. Mientras los primeros sistemas se limitaban al \emph{prompting} y la generación pasiva de texto, la investigación avanza hacia arquitecturas capaces de razonar, planificar, colaborar y actuar mediante \emph{tool calling}, memoria operativa y coordinación multiagente. Este giro habilita sistemas activos con mayor control, interpretabilidad y capacidad de integración con servicios externos. \cite{Masterman2024Survey}

El estudio de referencia \emph{The Landscape of Emerging AI Agent Architectures for Reasoning, Planning, and Tool Calling: A Survey} propone una taxonomía que distingue entre arquitecturas de \emph{agente único} y \emph{multiagente}, analizando ventajas y limitaciones de cada enfoque. Los agentes únicos favorecen la consistencia y el control global del \emph{workflow}, pero presentan límites en paralelismo y especialización; los sistemas multiagente, organizados horizontalmente (colaboración entre pares) o verticalmente (liderazgo/orquestación), facilitan la división de subtareas, la evaluación cruzada y la reducción de carga cognitiva individual. \cite{Masterman2024Survey}

De forma transversal, la literatura identifica tres componentes críticos del desempeño: (i) \textit{razonamiento y planificación explícita}; (ii) \textit{tool calling} para ampliar las capacidades más allá del modelo; y (iii) \textit{memoria y retroalimentación} (humana o agentiva) para iterar sobre errores y consolidar conocimiento. Estas piezas han demostrado mejorar la robustez y la eficiencia en dominios abiertos y tareas con información incompleta. \cite{Masterman2024Survey}

Entre los hallazgos empíricos, destacan los equipos multiagente con liderazgos definidos y filtros de comunicación, que logran mayor eficiencia computacional en flujos complejos; asimismo, los esquemas dinámicos—donde agentes se incorporan o retiran según la fase—aportan flexibilidad frente a dominios cambiantes. Tales propiedades motivan arquitecturas distribuidas y adaptativas en aplicaciones reales. \cite{Masterman2024Survey}

\section{Aplicaciones y tendencias actuales}

Las aplicaciones abarcan planificación automática, simulación social, resolución colaborativa de problemas y automatización de \emph{workflows} cognitivos. La integración con LLMs habilita entornos cooperativos donde los agentes intercambian evidencia, validan hipótesis y actúan sobre fuentes de datos o APIs. En este contexto, el patrón \emph{ReAct} (Reflexión--Acción--Observación) se ha consolidado para resolver consultas complejas: el agente planifica, ejecuta herramientas (buscadores, bases de datos) y contrasta resultados antes de responder, mejorando interpretabilidad y reduciendo alucinaciones frente a enfoques puramente generativos. \cite{Yao2022ReAct}

En ingeniería de soluciones, los \emph{agentic design patterns} propuestos por guías de arquitectura en la nube sirven para seleccionar estructuras de orquestación. En particular, el \textbf{Patrón de Lógica Personalizada} resulta adecuado cuando el flujo exige bifurcaciones condicionales, decisiones en tiempo de ejecución e integración con múltiples herramientas, aceptando el costo de mayor complejidad para ganar control y trazabilidad. \cite{google-cloud-agentic-patterns}

\section{Orientación del presente trabajo}

A partir de la revisión, se identifica una brecha: abundan demostraciones conceptuales de coordinación multiagente, pero son menos comunes las integraciones con procesos reales de captura, validación y persistencia de datos. Este proyecto propone una arquitectura multiagente aplicada al \textbf{procesamiento documental tributario} con dos rutas: (i) \emph{pipeline determinístico} de ingesta (OCR/parseo, validación SUNAT, clasificación tributaria, persistencia) y (ii) \emph{ruta de consultas} con ReAct sobre un ORM.

Para documentos \emph{escaneados} o imágenes, se utiliza \textbf{PaddleOCR} (PP-OCRv5 y PP-StructureV3) por su cobertura multilingüe y parsers de estructura; para comprobantes electrónicos (PDF con texto embebido) se prioriza \textbf{PyMuPDF} para extraer texto sin OCR, reduciendo latencia y errores. La persistencia se implementa con \textbf{SQLAlchemy} y \textbf{SQLite} en el MVP (propiedades \textsc{ACID} y despliegue embebido), con camino de migración a PostgreSQL en producción. Esta integración práctica—percepción (OCR/PDF) + reglas tributarias + persistencia local—busca optimizar eficiencia, trazabilidad y auditabilidad del flujo completo. \cite{Cui2025PaddleOCR,PyMuPDFDocs,SQLAlchemy2Docs,SQLiteACID}

En el frente normativo peruano, la clasificación y deducibilidad de gastos se sustentan en lineamientos oficiales para personas naturales (topes, porcentajes por rubro y requisitos formales del comprobante), lo que condiciona las reglas del motor de deducción y las validaciones de la capa de dominio. \cite{gobpe-gastos-deducibles,sunat-consideraciones-deducibles}

\section{Conclusiones}

El estado del arte muestra una evolución rápida hacia sistemas \emph{agentic} con razonamiento explícito, coordinación multiagente y \emph{tool calling}, con beneficios en control, interpretabilidad y desempeño. Persisten retos en gestión de contexto compartido, comunicación eficiente y acoplamiento con sistemas operativos (OCR, validadores, bases de datos). El trabajo propuesto contribuye con una integración end-to-end centrada en documentos tributarios, alineando buenas prácticas de arquitectura con requisitos regulatorios y de calidad de datos.
