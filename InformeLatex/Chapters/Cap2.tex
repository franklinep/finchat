\chapter{Estado del arte}
\label{cap:estado_arte}

\section{Arquitecturas basadas en modelos de lenguaje y agentes}

El surgimiento de los modelos de lenguaje de gran tamaño ha llevado al desarrollo de arquitecturas agentic, en las que uno o varios agentes coordinan razonamiento, planificación y uso de herramientas externas para resolver tareas complejas. El survey de \textcite{masterman2024landscape} describe un panorama de arquitecturas de agente único y multiagente, destacando patrones como bucles de planificación–ejecución–reflexión, equipos de agentes especializados y el uso intensivo de herramientas (APIs, bases de datos, buscadores) para mitigar alucinaciones y mejorar la trazabilidad.

Entre los patrones de control más influyentes se encuentra ReAct (\textit{Reasoning + Acting}), propuesto por \textcite{yao2023react}. En este enfoque, el modelo genera de forma intercalada trazas de razonamiento en lenguaje natural y acciones concretas (por ejemplo, invocaciones de herramientas o APIs). Las observaciones retornadas por esas herramientas se incorporan en pasos posteriores del razonamiento, permitiendo ciclos iterativos de percepción–acción. Los experimentos de ReAct muestran que esta combinación reduce alucinaciones y mejora la interpretabilidad frente a técnicas basadas solo en cadenas de pensamiento.

Sobre esta base, trabajos recientes proponen arquitecturas multiagente en las que cada agente adopta un rol específico (por ejemplo, recuperación de contexto, redacción, verificación de hechos, planificación), coordinados mediante protocolos de comunicación explícitos. \textcite{masterman2024landscape} identifican patrones recurrentes como agentes supervisores, equipos de colaboración y roles especializados para razonamiento, acción y evaluación, ofreciendo criterios para seleccionar el tipo de arquitectura en función del problema a resolver.

\section{Frameworks para la orquestación de agentes}

Diversos frameworks han surgido para facilitar la construcción y orquestación de agentes basados en modelos de lenguaje. LangGraph se presenta como una librería orientada a representar flujos de trabajo como grafos dirigidos, donde los nodos corresponden a funciones (incluyendo agentes basados en LLM) y los edges controlan el paso de mensajes y la transición entre estados \parencite{langgraph2025graphapi}. Esta aproximación facilita la definición de flujos complejos, con ramificaciones, retroalimentación y paralelización, adecuados para sistemas con múltiples agentes y decisiones condicionales.

Por otro lado, Agno se posiciona como un framework para construir y ejecutar agentes con un enfoque en el rendimiento y la integración directa con aplicaciones web existentes. Su componente AgentOS expone endpoints FastAPI listos para producción y permite definir tanto equipos de agentes como flujos secuenciales paso a paso, con soporte para memoria, herramientas y almacenamiento del estado de las conversaciones \parencite{agno2025framework}. Una característica relevante para contextos sensibles es que AgentOS se ejecuta íntegramente en la infraestructura del desarrollador, sin enviar datos a servicios externos, lo que refuerza la privacidad.

En esta tesis se adopta Agno debido a que el problema a resolver puede modelarse mediante un flujo secuencial relativamente estable (ingesta, parseo, validación, clasificación y persistencia), para el cual un grafo complejo de nodos y edges aportaría poca ventaja adicional frente al aumento de complejidad. Además, la integración con FastAPI y la ejecución privada se alinean con la necesidad de proteger datos tributarios sensibles.

\section{OCR y extracción de información en comprobantes de pago}

La extracción precisa de texto y estructura de comprobantes de pago es un componente crítico de sistemas que buscan automatizar tareas tributarias o financieras. PaddleOCR es una de las bibliotecas de código abierto más utilizadas para OCR en documentos en múltiples idiomas. En su reporte técnico más reciente, \textcite{cui2025paddleocr30technicalreport} describen la familia PP-OCR, que incluye modelos ligeros optimizados para despliegue en dispositivos con recursos limitados y escenarios de producción en tiempo real.

PP-OCRv5 introduce mejoras en precisión y velocidad respecto a versiones anteriores, manteniendo un tamaño de modelo reducido y soportando flujos end-to-end para detección y reconocimiento de texto en imágenes. En contraste, módulos como PP-StructureV3 están orientados a la comprensión de la estructura de documentos complejos (tablas, formularios, layouts) y PaddleOCR-VL se enfoca en tareas multimodales de comprensión de documentos. Para el caso de boletas y facturas de consumo, donde el formato suele ser simple y el objetivo principal es recuperar texto, PP-OCRv5 ofrece un balance adecuado entre exactitud y eficiencia.

En documentos PDF nativos (por ejemplo, boletas electrónicas descargadas del portal de SUNAT), el texto se encuentra embebido en el documento y puede extraerse de forma directa, sin necesidad de OCR. Herramientas como PyMuPDF permiten recuperar texto, imágenes y metadatos con gran velocidad y bajo consumo de recursos \parencite{pymupdf2025,artifex2025pymupdf}. Estudios y guías de buenas prácticas recomiendan una estrategia híbrida: intentar primero la extracción nativa y recurrir al OCR únicamente cuando la página contiene solo imágenes o texto ilegible, dado que el OCR completo puede requerir tiempos de cómputo cien o mil veces mayores que la extracción directa.

\section{Aplicaciones de IA en el dominio tributario y financiero}

En el dominio tributario, la mayor parte de las iniciativas tecnológicas se ha orientado al desarrollo de portales web, asistentes de llenado de declaraciones y sistemas de consulta de gastos deducibles provistos por la administración tributaria o por entidades privadas. SUNAT, por ejemplo, ofrece plataformas en línea que permiten consultar los gastos deducibles registrados y los topes de deducción aplicables a las rentas de trabajo \parencite{sunat2025gastos3uit}. Sin embargo, estas herramientas suelen requerir que la persona usuaria se adapte a interfaces estructuradas y no están diseñadas para recibir consultas en lenguaje natural ni para integrar comprobantes físicos escaneados.

En el ámbito privado, se han desarrollado aplicaciones de control de gastos personales y de planificación financiera que permiten registrar movimientos y generar reportes agregados. No obstante, la mayoría de estas soluciones se basan en la introducción manual de datos o en la sincronización con estados de cuenta bancarios, y no aprovechan de manera integrada técnicas modernas como OCR, modelos de lenguaje y arquitecturas multiagente para interpretar comprobantes tributarios y aplicar reglas de deducibilidad.

Los trabajos específicos que combinan OCR, modelos de lenguaje y validación tributaria con datos oficiales son todavía incipientes. Si bien existen proyectos experimentales que aplican LLMs para responder preguntas sobre normativa, estos suelen centrarse en la recuperación de fragmentos normativos (enfoque RAG) y no en la trazabilidad de comprobantes individuales ni en la integración con fuentes oficiales como el estado del RUC o las actividades económicas registradas.

\section{Síntesis crítica y brecha identificada}

A partir de la revisión efectuada se puede sintetizar lo siguiente:

\begin{itemize}
  \item Las arquitecturas agentic basadas en modelos de lenguaje han demostrado ser efectivas para coordinar razonamiento y uso de herramientas, especialmente cuando se emplean patrones como ReAct y equipos de agentes especializados.
  \item Existen frameworks maduros para orquestar agentes (LangGraph, Agno, entre otros), con distintos compromisos entre expresividad, complejidad y facilidad de integración con aplicaciones web existentes.
  \item Las soluciones de OCR actuales permiten extraer texto con alta precisión tanto en imágenes como en PDFs nativos, destacando las familias de modelos ligeros como PP-OCRv5 para escenarios en los que la latencia y el consumo de recursos son críticos.
  \item En el dominio tributario peruano predominan herramientas que permiten consultar gastos deducibles y presentar declaraciones, pero con interfaces estructuradas y sin integración con agentes conversacionales ni con comprobantes físicos escaneados.
\end{itemize}

Sin embargo, los trabajos revisados no abordan de manera integrada la combinación de:

\begin{enumerate}
  \item Una arquitectura conversacional multiagente para ingesta, validación y consulta de comprobantes tributarios.
  \item Un pipeline híbrido de extracción (PyMuPDF + PP-OCRv5) adaptado a boletas electrónicas y físicas.
  \item Herramientas SQL deterministas controladas por un agente ReAct para responder consultas en lenguaje natural sobre gastos y deducciones.
  \item Una aplicación móvil que sirva de interfaz principal para la persona usuaria, preservando la privacidad mediante el uso de modelos locales cuando es posible.
\end{enumerate}

Esta brecha motiva la propuesta planteada en los capítulos siguientes, en donde se diseña e implementa una arquitectura multiagente orientada específicamente a la gestión de gastos deducibles de personas naturales en el Perú.
