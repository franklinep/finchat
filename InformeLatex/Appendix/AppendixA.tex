% =========================
% APÉNDICES
% =========================
\appendix

\chapter{Mapa normativo de deducibilidad SUNAT}
\label{app:normativa-sunat}

Este apéndice resume los porcentajes y el tope adicional de deducción para personas naturales, con referencias oficiales.
Los porcentajes rigen dentro del límite de \textbf{3 UIT (S/ 16\,050 en 2025)}. \cite{gobpe-gastos-deducibles, sunat-consideraciones-deducibles}

\begin{table}[H]
\centering
\small
\begin{tabular}{p{6cm} p{3cm} p{7cm}}
\toprule
\textbf{Rubro} & \textbf{Porcentaje} & \textbf{Notas / Fuente}\\
\midrule
Restaurantes y hoteles & 15\% & Porcentajes y tope 3 UIT. \cite{gobpe-gastos-deducibles}\\
Servicios profesionales (4ta) & 30\% & \cite{gobpe-gastos-deducibles}\\
Alquiler de inmuebles & 30\% & \cite{gobpe-gastos-deducibles}\\
Aportaciones EsSalud (trabaj.\ del hogar) & 100\% & Dentro del tope 3 UIT. \cite{gobpe-gastos-deducibles}\\
\bottomrule
\end{tabular}
\caption{Resumen de porcentajes y tope adicional (3 UIT).}
\end{table}

\noindent\textit{Requisitos formales del comprobante}: identificación correcta del contribuyente en el comprobante, emisión válida, etc. \cite{sunat-consideraciones-deducibles}

\chapter{Glosario y abreviaturas}
\label{app:glosario}

\begin{description}
  \item[OCR] Reconocimiento Óptico de Caracteres.
  \item[ORM] \textit{Object–Relational Mapping}.
  \item[ReAct] Patrón \textit{Reflexión–Acción–Observación}.
  \item[UIT] Unidad Impositiva Tributaria.
\end{description}
