\documentclass[12pt,spanish, singlespacing,]{MastersDoctoralThesis}
\usepackage[utf8]{inputenc}
\usepackage[T1]{fontenc}
\usepackage[none]{hyphenat}
\usepackage{palatino}
\usepackage{acronym}
\spacing{1.5}
\usepackage[acronym,nomain]{glossaries}
%\usepackage[backend=bibtex,style=authoryear,natbib=true]{biblatex}
%\addbibresource{main.bib}
\usepackage[autostyle=true]{csquotes}
\usepackage{pdflscape,booktabs}
\usepackage{afterpage}
\newcommand\blankpage{%
    \null
    \thispagestyle{empty}%
    \addtocounter{page}{0}%
    \newpage}

\usepackage{listings}
\renewcommand{\lstlistingname}{Código}
\renewcommand\lstlistlistingname{Índice de Código}

\usepackage{float}
\usepackage{url}
\makeatletter
\g@addto@macro{\UrlBreaks}{\UrlOrds}
\makeatother

\thesistitle{}
\supervisor{CESAR JESUS LARA AVILA}
\examiner{}
\degree{}
\author{FRANKLIN ESPINOZA PARI}
\subject{}
\keywords{}
\university{\href{http://www.uni.edu.pe/}{Universidad Nacional de Ingenier\'ia}}
\department{\href{http://fc.uni.edu.pe/fc/index.php/escuelas/ciencia-de-la-computacion}{}}
\faculty{\href{http://fc.uni.edu.pe/fc/}{}}
\hypersetup{pdftitle=\ttitle}
\hypersetup{pdfauthor=\authorname}
\hypersetup{pdfkeywords=\keywordnames}
\sloppy
\decimalpoint
\begin{document}
\frontmatter
\pagestyle{plain}
\begin{titlepage}
\begin{center}
%\textsc{\huge \univname}\\[0.5cm]

\begin{figure}[h]
\centering
\includegraphics[width=0.3\textwidth]{Figures/Escudo_UNI.jpg}
\end{figure}

\textsc{\huge \univname}\\[0.3cm]
\textsc{\Large Facultad de Ciencias}\\[0.2cm]
\textsc{\large Escuela Profesional de Ciencia de la Computaci\'on}\\[2cm]
\textsc{\LARGE \textit{Sistema Conversacional Multiagente para la Automatización de Deducciones Tributarias en Perú mediante OCR y Validación SUNAT}}\\[1cm]
{\Large \textbf{SEMINARIO DE TESIS II}}
{\huge \bfseries \ttitle}\\[2cm]
% Thesis title
%\bigskip
%\HRule \\[0.8cm] % Horizontal line

%\begin{minipage}{1.5\textwidth}
%\begin{flushleft} \large
\bigskip
\bigskip
\large\emph{Autor:}
{\authorname}\\
\large\emph{Asesor:}
{\supname}
%\end{flushleft}
%\end{minipage}
%\\[2cm]
\\[1cm]
{\large OCTUBRE, 2025}\\[4cm]

\vfill
\end{center}
\end{titlepage}
\afterpage{\blankpage}
%\cleardoublepage
%\renewcommand{\abstractname}{Abstract}
\begin{abstract}
\addchaptertocentry{\abstractname}

El presente trabajo propone el diseño e implementación de un sistema conversacional multiagente para automatizar el registro y la validación de gastos deducibles en el contexto tributario peruano. El problema aborda la captura de información desde comprobantes heterogéneos (PDF e imágenes), la normalización de campos relevantes (RUC, tipo/serie/número, fecha, monto y emisor), la detección de duplicados y la verificación contra fuentes oficiales de la SUNAT, tareas que hoy suelen ejecutarse de forma manual, con altos costos de tiempo y propensión a errores.

La solución se estructura en una arquitectura basada en servicios: un agente de ingesta con OCR que extrae texto de documentos (PDF/imágenes) y lo transforma en unidades semánticas; un agente de deduplicación que previene el registro redundante; un agente de validación que consulta los servicios de SUNAT para verificar la consistencia y vigencia de los comprobantes; y un orquestador conversacional que guía al usuario, explica decisiones y solicita datos faltantes. El backend expone un webhook para recepción de archivos, persistencia transaccional de artefactos y trazabilidad de cada trabajo, con un modelo de datos que separa el blob del archivo y los fragmentos OCR para facilitar auditoría y recuperación.

La metodología de desarrollo fue incremental, organizada en sprints semanales, con énfasis en pruebas unitarias e integración para asegurar idempotencia, manejo de errores y repetibilidad. Como resultado, se obtuvo un prototipo funcional capaz de: (i) recibir documentos desde un canal conversacional, (ii) realizar OCR y almacenamiento estructurado, (iii) detectar duplicados de forma determinista, y (iv) preparar la validación con SUNAT. Se discuten decisiones de diseño, consideraciones legales y de privacidad para el tratamiento de datos tributarios, y un plan de evaluación que contempla precisión de OCR, tasa de falsos duplicados, latencia extremo a extremo y exactitud de verificación frente a SUNAT. Finalmente, se delinean extensiones futuras: interfaz de chat para usuarios finales, enriquecimiento con RAG, y despliegue seguro en producción.

\textbf{Palabras clave:} OCR, validación tributaria, SUNAT, multiagente, sistemas conversacionales, RAG, deducciones, Perú, FastAPI.
\end{abstract}
\afterpage{\blankpage}
\tableofcontents
%\afterpage{\blankpage}
\listoffigures
%\afterpage{\blankpage}
%\listoftables
%\afterpage{\blankpage}
%\listoftables
%\lstlistoflistings
%\afterpage{\blankpage}
\newpage
\begin{center}
{\huge Índice de Acrónimos}\\[2cm]
\end{center}
\bigskip
\begin{tabular}{ l c l }
\textbf{API} & & Application Programming Interface\\
\textbf{AJAX} & & Asynchronous JavaScript And XML\\
\textbf{ETC} & & Etcétera \\

\end{tabular}
\afterpage{\blankpage}

%\newpage
%\begin{center}
%{\huge \textit{Agradecmientos}}\\[1.5cm]
%\end{center}

%The acknowledgements and the people to thank go here, don't forget to include your project advisor\ldots
%With thanks to my friends, my family, my cats and Chris

\afterpage{\blankpage}

\mainmatter
\pagestyle{thesis}
\chapter{Introducción}

En el contexto peruano, tanto los trabajadores independientes como quienes laboran en relación de dependencia pueden deducir determinados gastos de sus rentas de trabajo, reduciendo así su carga tributaria anual. De acuerdo con la normativa vigente de la Superintendencia Nacional de Aduanas y de Administración Tributaria (SUNAT), los principales gastos deducibles son:

\begin{enumerate}
    \item \textbf{Hoteles y restaurantes:} 15\% del gasto efectuado.
    \item \textbf{Servicios profesionales:} 30\% del gasto efectuado.
    \item \textbf{Alquiler de inmuebles:} 30\% del gasto efectuado.
    \item \textbf{Aportes a EsSalud por trabajadores del hogar:} 100\% del gasto efectuado.
\end{enumerate}

Estos beneficios constituyen una oportunidad para los contribuyentes que buscan optimizar su economía personal mediante una adecuada gestión de sus deducciones tributarias. No obstante, en la práctica, muchas personas desconocen cuáles de sus gastos son efectivamente deducibles o cuánto han acumulado a lo largo del año fiscal. Ello se debe, principalmente, a que no todas las boletas se registran de manera automática en la plataforma de la SUNAT—en especial las emitidas en formato físico—y a dificultades en la administración de comprobantes de pago, la heterogeneidad de formatos y los errores derivados del procesamiento manual de información.

Frente a esta problemática, el presente trabajo propone el diseño e implementación de un sistema experimental para la \textbf{gestión automatizada de boletas y comprobantes}, clasificándolos según las categorías establecidas por la SUNAT y las condiciones necesarias para que el gasto sea deducible. El sistema integrará tecnologías de \textit{Reconocimiento Óptico de Caracteres} (OCR) mediante \textbf{PaddleOCR}, junto con un modelo de base de datos local gestionado con un \textbf{ORM} (mapeo objeto–relacional) sobre \textbf{SQLite}. Asimismo, se adoptará una arquitectura \textbf{multiagente}, en la que agentes especializados colaboran para ejecutar tareas específicas, tales como la lectura, la clasificación y la explicación de los datos obtenidos.

Adicionalmente, el sistema permitirá la generación automática de reportes, paneles de control y métricas de deducción, brindando al usuario una visión integral de sus gastos deducibles. De este modo, se busca facilitar el cumplimiento tributario y promover una mejor organización financiera personal.

\section{Motivación}

La motivación de este proyecto surge de una necesidad frecuente entre los contribuyentes: la ausencia de herramientas tecnológicas que automaticen y sistematicen la gestión de comprobantes de pago. En el Perú, una parte importante de contribuyentes no aprovecha plenamente los beneficios de las deducciones tributarias por desconocimiento, desorganización documental o falta de sistemas digitales de apoyo.

Si bien la normativa establece el registro electrónico de las boletas en la SUNAT, en la práctica muchas boletas físicas no llegan a cargarse. Aquellas no registradas deben declararse manualmente al cierre del ejercicio fiscal, lo que genera incertidumbre sobre el monto real de las deducciones acumuladas y, con frecuencia, pérdidas económicas por deducciones no declaradas oportunamente.

A partir de experiencias personales y familiares, se ha identificado la dificultad de conservar boletas físicas y digitales en buen estado. La falta de sistematización documental impacta directamente en la economía de los contribuyentes, al limitar su capacidad de aprovechar los beneficios tributarios disponibles.

En respuesta, un sistema automatizado basado en inteligencia artificial y visión computacional constituye una alternativa viable. Tecnologías como \textbf{PaddleOCR} permiten digitalizar la información contenida en documentos escaneados con alta precisión; el uso de un \textbf{ORM} con \textbf{SQLite} ofrece un entorno eficiente para almacenar y consultar datos estructurados de forma local; y una \textbf{arquitectura multiagente} facilita distribuir funciones en componentes autónomos y colaborativos, mejorando la escalabilidad y el mantenimiento.

Este trabajo busca demostrar la viabilidad de un enfoque multiagente aplicado a la gestión automatizada de comprobantes tributarios, combinando técnicas de reconocimiento de texto, procesamiento estructurado de datos y generación de reportes dinámicos.

\section{Objetivos}

Diseñar e implementar una arquitectura conversacional multiagente que automatice la captura, validación y análisis de comprobantes tributarios deducibles en el Perú, comparando el desempeño de distintos frameworks agentic (LangGraph y Agno), motores de extracción de texto (OCR local con modelos en Ollama frente a PaddleOCR y PyMuPDF) y modelos de lenguaje con diferentes cantidades de parámetros, en el contexto de un aplicativo móvil para usuarios finales.

\subsection*{Objetivos específicos del sistema}
\begin{itemize}
    \item Diseñar una arquitectura multiagente que combine agentes secuenciales para la coordinacion de la ingesta y validación, y agentes ReAct para la consulta del usuario.
    \item Diseñar e implementar una base de datos local con PostgreSQL, utilizando un ORM para la gestión eficientes de los datos estructurados extraidos.
    \item Definir y generarar datasets sintéticos y reales etiquetados para métricas cuantitativas.
    \item Evaluar motores de extracción de texto (OCR local con Ollama frente a PaddleOCR y PyMuPDF).
    \item Analizar el rendimiento de los modelos de lengauje de distinto tamaño para una aplicación móvil.
\end{itemize}

\subsection*{Objetivos académicos}
\begin{itemize}
    \item Aplicar conocimientos de arquitectura multiagente, ORM y bases de datos relacionales en un contexto real.
    \item Analizar el rendimiento de los motores de extracción de texto y modelos de lenguaje en un contexto real.
\end{itemize}

\section{Estructura del seminario}

Con el propósito de brindar una visión global del contenido, a continuación se presenta la estructura de los capítulos que componen la tesis:

\begin{itemize}
    \item \textbf{Introducción:} Contexto, problemática, motivación y objetivos del proyecto, además de una descripción general de la propuesta.
    \item \textbf{Estado del Arte:} Revisión de investigaciones y sistemas previos relacionados con el reconocimiento de texto, la automatización de procesos tributarios y las arquitecturas multiagente.
    \item \textbf{Marco Teórico:} Fundamentos del reconocimiento óptico de caracteres (OCR), arquitecturas multiagente, ORM y bases de datos relacionales.
    \item \textbf{Metodología y Herramientas:} Arquitectura del sistema, fases de desarrollo, tecnologías empleadas y aspectos éticos vinculados al manejo de datos personales.
    \item \textbf{Resultados y discusión:} Presentación de resultados experimentales, incluyendo métricas de precisión y desempeño en el reconocimiento y la categorización de comprobantes reales.
    \item \textbf{Conclusiones y Trabajo Futuro:} Principales hallazgos y posibles extensiones del sistema, como la integración con plataformas oficiales y el desarrollo de una interfaz web completa.
\end{itemize}

%\newpage
%$\ $
%\thispagestyle{empty} % para que no se numere esta pagina
\chapter{Estado del arte}
\label{cap:estado_arte}

\section{Arquitecturas basadas en modelos de lenguaje y agentes}

El surgimiento de los modelos de lenguaje de gran tamaño ha llevado al desarrollo de arquitecturas agentic, en las que uno o varios agentes coordinan razonamiento, planificación y uso de herramientas externas para resolver tareas complejas. El survey de \textcite{masterman2024landscape} describe un panorama de arquitecturas de agente único y multiagente, destacando patrones como bucles de planificación–ejecución–reflexión, equipos de agentes especializados y el uso intensivo de herramientas (APIs, bases de datos, buscadores) para mitigar alucinaciones y mejorar la trazabilidad.

Entre los patrones de control más influyentes se encuentra ReAct (\textit{Reasoning + Acting}), propuesto por \textcite{yao2023react}. En este enfoque, el modelo genera de forma intercalada trazas de razonamiento en lenguaje natural y acciones concretas (por ejemplo, invocaciones de herramientas o APIs). Las observaciones retornadas por esas herramientas se incorporan en pasos posteriores del razonamiento, permitiendo ciclos iterativos de percepción–acción. Los experimentos de ReAct muestran que esta combinación reduce alucinaciones y mejora la interpretabilidad frente a técnicas basadas solo en cadenas de pensamiento.

Sobre esta base, trabajos recientes proponen arquitecturas multiagente en las que cada agente adopta un rol específico (por ejemplo, recuperación de contexto, redacción, verificación de hechos, planificación), coordinados mediante protocolos de comunicación explícitos. \textcite{masterman2024landscape} identifican patrones recurrentes como agentes supervisores, equipos de colaboración y roles especializados para razonamiento, acción y evaluación, ofreciendo criterios para seleccionar el tipo de arquitectura en función del problema a resolver.

\section{Frameworks para la orquestación de agentes}

Diversos frameworks han surgido para facilitar la construcción y orquestación de agentes basados en modelos de lenguaje. LangGraph se presenta como una librería orientada a representar flujos de trabajo como grafos dirigidos, donde los nodos corresponden a funciones (incluyendo agentes basados en LLM) y los edges controlan el paso de mensajes y la transición entre estados \parencite{langgraph2025graphapi}. Esta aproximación facilita la definición de flujos complejos, con ramificaciones, retroalimentación y paralelización, adecuados para sistemas con múltiples agentes y decisiones condicionales.

Por otro lado, Agno se posiciona como un framework para construir y ejecutar agentes con un enfoque en el rendimiento y la integración directa con aplicaciones web existentes. Su componente AgentOS expone endpoints FastAPI listos para producción y permite definir tanto equipos de agentes como flujos secuenciales paso a paso, con soporte para memoria, herramientas y almacenamiento del estado de las conversaciones \parencite{agno2025framework}. Una característica relevante para contextos sensibles es que AgentOS se ejecuta íntegramente en la infraestructura del desarrollador, sin enviar datos a servicios externos, lo que refuerza la privacidad.

En esta tesis se adopta Agno debido a que el problema a resolver puede modelarse mediante un flujo secuencial relativamente estable (ingesta, parseo, validación, clasificación y persistencia), para el cual un grafo complejo de nodos y edges aportaría poca ventaja adicional frente al aumento de complejidad. Además, la integración con FastAPI y la ejecución privada se alinean con la necesidad de proteger datos tributarios sensibles.

\section{OCR y extracción de información en comprobantes de pago}

La extracción precisa de texto y estructura de comprobantes de pago es un componente crítico de sistemas que buscan automatizar tareas tributarias o financieras. PaddleOCR es una de las bibliotecas de código abierto más utilizadas para OCR en documentos en múltiples idiomas. En su reporte técnico más reciente, \textcite{cui2025paddleocr30technicalreport} describen la familia PP-OCR, que incluye modelos ligeros optimizados para despliegue en dispositivos con recursos limitados y escenarios de producción en tiempo real.

PP-OCRv5 introduce mejoras en precisión y velocidad respecto a versiones anteriores, manteniendo un tamaño de modelo reducido y soportando flujos end-to-end para detección y reconocimiento de texto en imágenes. En contraste, módulos como PP-StructureV3 están orientados a la comprensión de la estructura de documentos complejos (tablas, formularios, layouts) y PaddleOCR-VL se enfoca en tareas multimodales de comprensión de documentos. Para el caso de boletas y facturas de consumo, donde el formato suele ser simple y el objetivo principal es recuperar texto, PP-OCRv5 ofrece un balance adecuado entre exactitud y eficiencia.

En documentos PDF nativos (por ejemplo, boletas electrónicas descargadas del portal de SUNAT), el texto se encuentra embebido en el documento y puede extraerse de forma directa, sin necesidad de OCR. Herramientas como PyMuPDF permiten recuperar texto, imágenes y metadatos con gran velocidad y bajo consumo de recursos \parencite{pymupdf2025,artifex2025pymupdf}. Estudios y guías de buenas prácticas recomiendan una estrategia híbrida: intentar primero la extracción nativa y recurrir al OCR únicamente cuando la página contiene solo imágenes o texto ilegible, dado que el OCR completo puede requerir tiempos de cómputo cien o mil veces mayores que la extracción directa.

\section{Aplicaciones de IA en el dominio tributario y financiero}

En el dominio tributario, la mayor parte de las iniciativas tecnológicas se ha orientado al desarrollo de portales web, asistentes de llenado de declaraciones y sistemas de consulta de gastos deducibles provistos por la administración tributaria o por entidades privadas. SUNAT, por ejemplo, ofrece plataformas en línea que permiten consultar los gastos deducibles registrados y los topes de deducción aplicables a las rentas de trabajo \parencite{sunat2025gastos3uit}. Sin embargo, estas herramientas suelen requerir que la persona usuaria se adapte a interfaces estructuradas y no están diseñadas para recibir consultas en lenguaje natural ni para integrar comprobantes físicos escaneados.

En el ámbito privado, se han desarrollado aplicaciones de control de gastos personales y de planificación financiera que permiten registrar movimientos y generar reportes agregados. No obstante, la mayoría de estas soluciones se basan en la introducción manual de datos o en la sincronización con estados de cuenta bancarios, y no aprovechan de manera integrada técnicas modernas como OCR, modelos de lenguaje y arquitecturas multiagente para interpretar comprobantes tributarios y aplicar reglas de deducibilidad.

Los trabajos específicos que combinan OCR, modelos de lenguaje y validación tributaria con datos oficiales son todavía incipientes. Si bien existen proyectos experimentales que aplican LLMs para responder preguntas sobre normativa, estos suelen centrarse en la recuperación de fragmentos normativos (enfoque RAG) y no en la trazabilidad de comprobantes individuales ni en la integración con fuentes oficiales como el estado del RUC o las actividades económicas registradas.

\section{Síntesis crítica y brecha identificada}

A partir de la revisión efectuada se puede sintetizar lo siguiente:

\begin{itemize}
  \item Las arquitecturas agentic basadas en modelos de lenguaje han demostrado ser efectivas para coordinar razonamiento y uso de herramientas, especialmente cuando se emplean patrones como ReAct y equipos de agentes especializados.
  \item Existen frameworks maduros para orquestar agentes (LangGraph, Agno, entre otros), con distintos compromisos entre expresividad, complejidad y facilidad de integración con aplicaciones web existentes.
  \item Las soluciones de OCR actuales permiten extraer texto con alta precisión tanto en imágenes como en PDFs nativos, destacando las familias de modelos ligeros como PP-OCRv5 para escenarios en los que la latencia y el consumo de recursos son críticos.
  \item En el dominio tributario peruano predominan herramientas que permiten consultar gastos deducibles y presentar declaraciones, pero con interfaces estructuradas y sin integración con agentes conversacionales ni con comprobantes físicos escaneados.
\end{itemize}

Sin embargo, los trabajos revisados no abordan de manera integrada la combinación de:

\begin{enumerate}
  \item Una arquitectura conversacional multiagente para ingesta, validación y consulta de comprobantes tributarios.
  \item Un pipeline híbrido de extracción (PyMuPDF + PP-OCRv5) adaptado a boletas electrónicas y físicas.
  \item Herramientas SQL deterministas controladas por un agente ReAct para responder consultas en lenguaje natural sobre gastos y deducciones.
  \item Una aplicación móvil que sirva de interfaz principal para la persona usuaria, preservando la privacidad mediante el uso de modelos locales cuando es posible.
\end{enumerate}

Esta brecha motiva la propuesta planteada en los capítulos siguientes, en donde se diseña e implementa una arquitectura multiagente orientada específicamente a la gestión de gastos deducibles de personas naturales en el Perú.

%\newpage
$\ $
%\thispagestyle{empty} % para que no se numere esta pagina
\chapter{Metodología y arquitectura propuesta}
\label{cap:metodologia}

\section{Enfoque metodológico de la investigación}

La presente investigación se enmarca en una \textit{investigación aplicada} con enfoque de diseño de artefactos software. El objetivo principal es construir y validar una arquitectura multiagente que permita gestionar comprobantes de pago y estimar gastos deducibles de manera conversacional.

Desde la perspectiva científica, se adopta un diseño cuasi-experimental: se construye el artefacto (sistema conversacional multiagente) y se evalúa su desempeño cuantitativo mediante métricas de precisión y latencia en tareas específicas (parseo, validación SUNAT y consulta). Los experimentos se realizan sobre un conjunto de comprobantes sintéticos y casos reales controlados, sin grupo de control basado en una arquitectura alternativa, dejando dicha comparación para trabajos futuros.

Desde la perspectiva de ingeniería, se sigue un proceso incremental compuesto por las siguientes etapas:

\begin{enumerate}
  \item Definición de requisitos funcionales y no funcionales, incluyendo restricciones tributarias, de seguridad y de experiencia de usuario.
  \item Diseño de la arquitectura general multiagente y del modelo de datos relacional.
  \item Implementación de los módulos de ingesta, OCR, agentes, backend y aplicación móvil.
  \item Diseño del protocolo experimental: construcción de datasets, definición de casos de prueba y métricas.
  \item Ejecución de experimentos y análisis de resultados.
\end{enumerate}

\section{Arquitectura general del sistema}

La arquitectura se organiza por funcionalidades (\textit{features}) y se compone de cuatro capas principales:

\begin{enumerate}
  \item \textbf{Capa de presentación}: aplicación móvil desarrollada con Expo/React Native, que expone interfaz de registro de comprobantes, vista de historial y chat con el asistente conversacional.
  \item \textbf{Capa de orquestación conversacional}: backend FastAPI que integra Agno (AgentOS) para definir y ejecutar agentes secuenciales y agentes basados en ReAct.
  \item \textbf{Capa de procesamiento de documentos}: módulo de OCR y extracción, que utiliza PyMuPDF para PDFs nativos y PP-OCRv5 para imágenes de boletas físicas.
  \item \textbf{Capa de persistencia}: base de datos PostgreSQL, accedida a través de repositorios que encapsulan el ORM y exponen una API orientada al dominio.
\end{enumerate}

En el contexto de Agno, el flujo de trabajo de ingesta se modela como un \textit{workflow} secuencial, mientras que el agente de consulta se implementa mediante el patrón ReAct, combinando razonamiento en lenguaje natural y llamadas a herramientas SQL deterministas.

En la tesis se recomienda incluir un diagrama de componentes que represente estas capas y su interacción, por ejemplo:

\begin{figure}[H]
  \centering
  \includegraphics[width=0.75\textwidth]{Figures/arquitectura_general.png}
  \caption{Arquitectura general del sistema conversacional multiagente.}
  \label{fig:arquitectura_general}
\end{figure}

\section{Arquitectura multiagente con Agno}

\subsection{Uso de Agno y justificación frente a LangGraph}

En la implementación se utiliza Agno como framework principal para la definición y ejecución de agentes. Agno provee un AgentOS que se integra de forma nativa con FastAPI, exponiendo endpoints para interactuar con agentes, gestionar sesiones, memoria y herramientas \parencite{agno2025framework}. Esta integración permite mantener una única aplicación backend coherente, donde los agentes conviven con el resto de endpoints de negocio.

LangGraph ofrece una abstracción basada en grafos para modelar aplicaciones con múltiples agentes y flujos complejos de decisión \parencite{langgraph2025graphapi}. Sin embargo, el flujo de ingesta y validación de comprobantes en este trabajo es esencialmente lineal y determinista: validar formato de archivo, aplicar OCR o extracción de texto, parsear, validar en SUNAT, clasificar y persistir. La complejidad adicional de un grafo de nodos y edges no se considera necesaria en esta etapa, y se prefiere una solución con flujos secuenciales explícitos y mayor cercanía a la lógica del dominio.

En síntesis, se elige Agno porque:

\begin{itemize}
  \item Permite definir \textit{workflows} secuenciales y equipos de agentes con bajo coste de configuración.
  \item Se integra de manera directa con FastAPI, facilitando la exposición de endpoints y el uso de middlewares de autenticación.
  \item Ofrece un modelo de ejecución \textit{private by design}, donde los agentes y sus datos corren íntegramente en la infraestructura controlada por el backend, protegiendo la información tributaria sensible.
\end{itemize}

\subsection{Agentes del workflow de ingesta}

El flujo de ingesta de comprobantes se implementa como una secuencia de agentes especializados:

\begin{enumerate}
  \item \textbf{Agente validador de comprobante}: verifica que el archivo subido sea un comprobante válido (tipo de archivo, número de páginas, tamaño, etc.) y genera un identificador de trabajo.
  \item \textbf{Agente de OCR y parseo}: aplica PyMuPDF o PP-OCRv5 según el tipo de documento, y luego utiliza un modelo de lenguaje para extraer un JSON estructurado con los campos del comprobante.
  \item \textbf{Agente validador SUNAT}: consulta el RUC del emisor mediante una herramienta externa (\texttt{consultar\_ruc}) y aplica reglas de deducibilidad basadas en el estado y condición del contribuyente y el código CIIU.
  \item \textbf{Agente clasificador}: asigna una categoría de gasto y un porcentaje de deducción aproximado en función de la normativa vigente y del CIIU detectado.
  \item \textbf{Agente de persistencia}: utiliza repositorios para almacenar comprobante, detalles, resultados de OCR, validaciones y clasificación en la base de datos.
\end{enumerate}

Estos agentes se conectan a través de un contexto compartido que encapsula el estado del trabajo de ingesta (identificador de usuario, comprobante, texto, resultados intermedios y errores). La secuencialidad favorece la trazabilidad, ya que cada etapa puede registrarse y auditarse por separado.

Para documentar este flujo, se recomienda incluir un diagrama de actividades o de flujo de procesos similar a:

\begin{figure}[H]
  \centering
  \includegraphics[width=0.75\textwidth]{Figures/flujo_ingesta.png}
  \caption{Flujo de trabajo de ingesta y validación de comprobantes.}
  \label{fig:flujo_ingesta}
\end{figure}

\subsection{Agente de consulta basado en ReAct}

El agente de consulta se implementa siguiendo el patrón ReAct \parencite{yao2023react}. En lugar de utilizar un esquema de Recuperación Aumentada (RAG) sobre embeddings, el agente trabaja con un conjunto de herramientas SQL deterministas que encapsulan consultas frecuentes sobre la base de datos, tales como:

\begin{itemize}
  \item Cálculo de montos totales por categoría y periodo.
  \item Listado de comprobantes por emisor o rango de fechas.
  \item Consultas sobre el porcentaje de deducción acumulado frente al tope de 3 UIT.
\end{itemize}

Dado un mensaje en lenguaje natural, el agente razona sobre la intención de la persona usuaria, selecciona la herramienta SQL apropiada y completa sus parámetros (por ejemplo, rango de fechas, categoría o emisor) en función del contenido de la consulta. El resultado de la herramienta se incorpora al contexto y el agente genera una respuesta final en lenguaje natural. Este enfoque evita complejidades asociadas al vectorizado del contenido y a la gestión de corpus extensos, aprovechando que los datos de interés se encuentran estructurados en la base de datos.

\section{Procesamiento de documentos: PyMuPDF y PP-OCRv5}

\subsection{Extracción de texto en boletas electrónicas con PyMuPDF}

Las boletas electrónicas descargadas de SUNAT comparten un formato PDF nativo en el que el texto se encuentra embebido de forma estructurada. Para este tipo de documentos, se utiliza PyMuPDF como motor de extracción, lo que permite recuperar el texto con alta velocidad y sin introducir errores de reconocimiento óptico de caracteres \parencite{pymupdf2025,artifex2025pymupdf}. Diversas evaluaciones independientes reportan que la extracción nativa con PyMuPDF es significativamente más rápida que métodos basados en OCR y que es adecuada para escenarios de alto volumen y baja latencia.

En el sistema propuesto, PyMuPDF se emplea como primera opción cuando se detecta que el PDF contiene texto embebido. Solo en caso de que la extracción falle o devuelva texto insuficiente se considera la posibilidad de aplicar OCR como respaldo. Esta estrategia reduce de manera importante el tiempo de procesamiento de boletas electrónicas y minimiza la propagación de errores de OCR hacia las etapas posteriores.

\subsection{OCR en boletas físicas con PP-OCRv5}

Para boletas físicas escaneadas o fotografiadas, se integra PP-OCRv5 como motor de OCR. Los modelos PP-OCRv5 se caracterizan por su tamaño reducido y por estar optimizados para despliegue en dispositivos y escenarios de producción donde la latencia es crítica \parencite{cui2025paddleocr30technicalreport}. La combinación de un detector de texto ligero con un reconocedor eficiente permite procesar imágenes de comprobantes con precisión suficiente para extraer campos como RUC, razón social, fecha y montos.

Se descartan, para este trabajo, modelos más pesados como PP-StructureV3 y PaddleOCR-VL, que se orientan a tareas de comprensión estructural o multimodal más complejas. Dado que las boletas de consumo presentan estructuras relativamente simples y que la prioridad es la velocidad, PP-OCRv5 ofrece un equilibrio adecuado entre rendimiento y consumo de recursos.

\section{Modelo de datos y patrón de repositorios}

\subsection{Modelo relacional en PostgreSQL}

El sistema utiliza PostgreSQL como gestor de base de datos relacional. El modelo de datos se centra en las siguientes entidades:

\begin{itemize}
  \item \textbf{Usuario}: almacena información básica de la persona usuaria y sus credenciales.
  \item \textbf{Emisor}: representa al contribuyente que emite el comprobante (RUC, razón social, nombre comercial, CIIU, estado y condición del RUC).
  \item \textbf{Comprobante}: contiene los datos principales del comprobante (tipo, serie, número, fecha, monto, moneda, origen, hashes de archivo, estado de procesamiento, flags de deducibilidad y duplicado).
  \item \textbf{Detalle\_comprobante}: almacena los ítems individuales del comprobante.
  \item \textbf{Validación}: registra los resultados de la consulta a SUNAT (estado y condición del RUC, CIIU detectado, coincidencia de nombres y regla de deducibilidad).
  \item \textbf{Clasificación}: guarda la categoría de gasto y el porcentaje de deducción asociado.
  \item \textbf{OCR\_pagina}: conserva el texto y una medida de confianza promedio para cada página OCR.
  \item \textbf{Estado\_trabajo}: gestiona el estado de los trabajos de ingesta o consulta.
  \item \textbf{Historial\_chat}: registra los mensajes intercambiados en la interfaz conversacional.
\end{itemize}

En la tesis se recomienda incluir un diagrama entidad–relación que represente estas tablas y sus relaciones, por ejemplo:

\begin{figure}[H]
  \centering
  \includegraphics[width=0.85\textwidth]{Figures/modelo_datos.png}
  \caption{Modelo entidad–relación de la base de datos del sistema.}
  \label{fig:modelo_datos}
\end{figure}

\subsection{Patrón de repositorios e inversión de dependencias}

Para desacoplar la lógica de negocio del ORM utilizado, se adopta el patrón de repositorios. Cada agregado principal (por ejemplo, \textit{Comprobante}, \textit{Emisor}, \textit{Usuario}) cuenta con una interfaz de repositorio que define operaciones orientadas al dominio (registrar un comprobante, buscar comprobantes por usuario y periodo, actualizar el resultado de validación, etc.). La implementación concreta de estas interfaces se realiza mediante el ORM, pero los agentes y servicios dependen únicamente de las abstracciones.

Este enfoque ofrece varias ventajas:

\begin{itemize}
  \item Facilita la prueba unitaria de la lógica de negocio, sustituyendo repositorios reales por dobles de prueba.
  \item Permite cambiar el ORM o incluso la tecnología de persistencia en el futuro sin modificar el código de los agentes.
  \item Expone operaciones semánticamente significativas, alineadas con el dominio tributario, en lugar de consultas SQL dispersas en el código.
\end{itemize}

\section{Backend y endpoints expuestos}

El backend se implementa con FastAPI y expone los siguientes endpoints principales:

\begin{itemize}
  \item \texttt{POST /api/v1/auth/register}: registra un nuevo usuario en el sistema, almacenando correo electrónico y contraseña en forma de hash.
  \item \texttt{POST /api/v1/auth/login}: autentica a la persona usuaria y entrega un token de acceso para invocar los demás endpoints.
  \item \texttt{POST /api/v1/comprobantes/subir}: recibe un comprobante en formato archivo (PDF o imagen) y dispara el workflow de ingesta, que ejecuta secuencialmente los agentes de validación de comprobante, OCR/parseo, validación SUNAT, clasificación y persistencia.
  \item \texttt{POST /api/v1/comprobantes/consultar}: recibe una consulta en lenguaje natural y delega al agente de consulta, el cual selecciona y ejecuta la herramienta SQL adecuada y devuelve una respuesta estructurada y una explicación en lenguaje natural.
\end{itemize}

Todos estos endpoints requieren autenticación y asocian la información procesada al usuario autenticado, garantizando aislamiento entre cuentas. La lógica de negocio se implementa en servicios y agentes, mientras que los controladores HTTP se limitan a gestionar validación básica de entrada, autenticación y manejo de errores.

\section{Aplicación móvil con Expo}

La capa de presentación se implementa como una aplicación móvil desarrollada con Expo sobre React Native. Expo provee un flujo administrado para construir aplicaciones multiplataforma (Android e iOS) desde una base de código única, con herramientas integradas para compilación en la nube, manejo de activos, notificaciones y actualizaciones OTA \parencite{expo2025ota,expo2025whyexpo}. Esto permite:

\begin{itemize}
  \item Reducir el tiempo de desarrollo y pruebas, ya que la mayoría de los cambios se concentran en el código JavaScript/TypeScript.
  \item Desplegar correcciones menores y mejoras en la interfaz mediante actualizaciones OTA, sin necesidad de pasar por todo el ciclo de revisión de las tiendas de aplicaciones para cada cambio.
  \item Integrar de manera sencilla la comunicación con el backend FastAPI, autenticación y almacenamiento local de sesiones.
\end{itemize}

La aplicación móvil implementa pantallas para el registro y autenticación de usuarios, la subida de comprobantes (seleccionando archivos o capturando fotografías) y un módulo de chat donde la persona usuaria puede realizar consultas en lenguaje natural sobre sus gastos y deducciones acumuladas.

\section{Conjunto de evaluación y métricas}

Para evaluar empíricamente el desempeño de la arquitectura propuesta se definen tres experimentos principales, cada uno alineado con un objetivo específico:

\begin{enumerate}
  \item \textbf{Evaluación del agente de parseo}: se construye un conjunto de 20 comprobantes sintéticos, en formato JSON, que contienen texto OCR simulado con variaciones en la forma de referirse al cliente y al emisor, así como en la estructura de los ítems. Se mide la latencia promedio de cada modelo, la tasa de éxito en la generación de JSON válido y la precisión en la extracción de campos clave.
  \item \textbf{Evaluación del agente validador SUNAT}: se diseñan 5 casos de prueba con RUC reales que cubren combinaciones de estado y condición del contribuyente, así como diferentes actividades económicas. Se evalúa la capacidad de los modelos para invocar correctamente la herramienta de consulta de RUC y para extraer campos como estado, condición y código CIIU.
  \item \textbf{Evaluación del agente de consulta}: se definen 4 consultas de prueba con diferentes niveles de complejidad (cálculo de totales, filtrado por fechas, búsqueda por emisor e inferencia de fechas relativas), y se mide la precisión en la selección de la herramienta SQL adecuada, la precisión en el llenado de argumentos y la latencia.
\end{enumerate}

Las métricas seleccionadas son:

\begin{itemize}
  \item \textbf{Latencia}: tiempo, en segundos, entre la recepción de la solicitud y la producción de la respuesta del agente.
  \item \textbf{Tasa de éxito}: proporción de ejecuciones en las que el formato de salida del agente es válido y procesable por las etapas siguientes (por ejemplo, JSON sintácticamente correcto).
  \item \textbf{Precisión de campos}: fracción de campos correctamente extraídos o inferidos respecto de la \textit{ground truth} definida para cada caso de prueba.
\end{itemize}

Los resultados de estos experimentos se presentan y discuten en el Capítulo~\ref{cap:resultados}.

%\newpage
$\ $
%\thispagestyle{empty} % para que no se numere esta pagina
\chapter{Marco teórico}
\label{cap:marco_teorico}

\section{Modelos de lenguaje de gran tamaño (LLM)}

Los modelos de lenguaje de gran tamaño (LLM, por sus siglas en inglés) son modelos neurales entrenados sobre grandes corpus de texto con el objetivo de predecir la siguiente palabra en una secuencia. A partir de esta tarea, aprenden representaciones distribuidas de palabras, frases y documentos que les permiten realizar tareas como generación de texto, traducción, resumen, razonamiento y respuesta a preguntas.

En el contexto de agentes conversacionales, los LLM no solo generan texto, sino que pueden utilizarse para controlar el flujo de una conversación, decidir qué herramientas invocar y combinar información de múltiples fuentes. El comportamiento de estos modelos se adapta mediante \textit{prompts} que describen el rol del agente, el formato esperado de salida y ejemplos de interacción, lo que resulta clave en esta tesis para obtener salidas estructuradas en formato JSON y para controlar el uso de herramientas.

\section{Arquitecturas agentic y patrón ReAct}

\subsection{Agentes basados en LLM}

Un agente basado en LLM puede definirse como un sistema que, apoyado en un modelo de lenguaje, percibe un contexto (mensaje del usuario, memoria, resultados de herramientas), decide una acción (responder, invocar una herramienta, delegar a otro agente), ejecuta dicha acción y actualiza su estado. El control del flujo puede ser implícito (mediante prompts) o explícito (mediante un framework que define estados, transiciones y memoria).

\textcite{masterman2024landscape} describen múltiples patrones de diseño para agentes y arquitecturas multiagente, incluyendo agentes supervisores, equipos horizontales, flujos de planificación–ejecución–reflexión y agentes especialistas por herramienta. Estas arquitecturas permiten descomponer problemas complejos en tareas más pequeñas, asignadas a agentes especializados.

\subsection{Patrón ReAct}

El patrón ReAct, propuesto por \textcite{yao2023react}, integra de forma explícita el razonamiento en lenguaje natural con la ejecución de acciones (uso de herramientas). Un ciclo típico ReAct incluye:

\begin{enumerate}
  \item \textbf{Razonamiento}: el modelo elabora una explicación parcial o plan en lenguaje natural sobre cómo resolver la tarea.
  \item \textbf{Acción}: el modelo decide invocar una herramienta concreta (por ejemplo, una API de búsqueda o una consulta SQL) con argumentos específicos.
  \item \textbf{Observación}: el sistema recibe la respuesta de la herramienta y la añade al contexto.
  \item \textbf{Iteración}: el modelo continúa razonando, incorporando la nueva información, hasta producir una respuesta final.
\end{enumerate}

Este patrón se utiliza en la tesis para el agente de consulta, que combina razonamiento sobre la intención del usuario con invocaciones a herramientas SQL deterministas. La salida final integra tanto la respuesta numérica o estructurada de la consulta como una explicación en lenguaje natural.

\section{OCR y extracción de información}

\subsection{Conceptos básicos de OCR}

El reconocimiento óptico de caracteres (OCR) es la técnica mediante la cual se convierte texto presente en imágenes (escaneos, fotografías) en texto digital editable. Un pipeline de OCR típico incluye etapas de preprocesamiento de la imagen (binarización, corrección de inclinación), detección de regiones de texto y reconocimiento de caracteres.

Las métricas comunes en OCR incluyen la tasa de error de caracteres (CER) y la tasa de error de palabras (WER). En el contexto de esta tesis, estas métricas no se miden directamente, pero influyen en la precisión de los campos extraídos por el agente de parseo.

\subsection{PaddleOCR y PP-OCRv5}

PaddleOCR es una librería de código abierto diseñada para proporcionar soluciones de OCR end-to-end en múltiples idiomas. Su familia de modelos PP-OCR se centra en ofrecer modelos ligeros con un buen equilibrio entre velocidad y precisión, adecuados para despliegue en producción y en dispositivos con recursos limitados \parencite{cui2025paddleocr30technicalreport}.

PP-OCRv5 integra mejoras en los módulos de detección y reconocimiento, así como optimizaciones en la arquitectura de red y en las técnicas de entrenamiento, lo que permite procesar imágenes más rápidamente manteniendo una precisión competitiva. Estas características lo convierten en una opción adecuada para el tratamiento de boletas físicas en un sistema que prioriza la latencia.

\subsection{Extracción de texto en PDF con PyMuPDF}

En documentos PDF nativos, el texto suele estar almacenado como entidades de texto internas en el archivo, lo que permite extraerlo sin recurrir a OCR. PyMuPDF es una librería que ofrece funciones de lectura y manipulación de PDFs, incluyendo métodos de extracción de texto, imágenes y metadatos con alta eficiencia \parencite{pymupdf2025,artifex2025pymupdf}. Guías de rendimiento reportan que la extracción de texto con PyMuPDF es varias veces más rápida que otras librerías y órdenes de magnitud más rápida que el OCR completo en documentos extensos.

En esta tesis se adopta una estrategia híbrida: se utiliza PyMuPDF para boletas electrónicas descargadas de la SUNAT y PP-OCRv5 solo cuando el comprobante está disponible únicamente como imagen.

\section{Conceptos tributarios relevantes}

\subsection{RUC, CIIU y condición del contribuyente}

En el Perú, toda persona natural o jurídica que realiza actividades económicas formales posee un Registro Único de Contribuyentes (RUC). El RUC incluye información como el estado del contribuyente (por ejemplo, ACTIVO o BAJA) y la condición (HABIDO o NO HABIDO), así como una o más actividades económicas clasificadas mediante códigos CIIU.

El estado y la condición del RUC constituyen insumos fundamentales para determinar si un comprobante de pago puede ser considerado deducible en términos tributarios: como regla simplificada, un emisor con RUC en estado ACTIVO y condición HABIDO se considera elegible para que sus comprobantes puedan sustentar gastos deducibles, mientras que ciertas combinaciones de estado o condición pueden conducir a la no deducibilidad.

\subsection{Gastos deducibles y deducción adicional de 3 UIT}

La normativa peruana establece que las personas naturales pueden deducir de la base imponible de su impuesto a la renta un conjunto de gastos personales hasta un límite dado, incluyendo una deducción adicional de hasta 3 UIT por consumos en rubros específicos como hoteles, restaurantes, alquiler de inmuebles, servicios profesionales y aportes a EsSalud por trabajadores del hogar \parencite{sunat2025gastos3uit}. Para que estos gastos sean deducibles, deben estar sustentados con comprobantes de pago electrónicos válidos y vinculados al contribuyente.

En este trabajo se implementa una capa de reglas simplificadas que, combinada con la información proveniente de la consulta al RUC, permite marcar un comprobante como potencialmente deducible o no deducible, con fines informativos para la persona usuaria.

\section{Métricas de evaluación}

Las métricas utilizadas en los experimentos responden a la necesidad de evaluar tanto la calidad de la información extraída como la eficiencia de la arquitectura:

\begin{itemize}
  \item \textbf{Latencia}: se define como el tiempo transcurrido entre la recepción de la solicitud y la generación de la respuesta por parte del agente, medido en segundos. Es relevante para la experiencia de usuario, especialmente en la aplicación móvil.
  \item \textbf{Tasa de éxito en el formato}: porcentaje de ejecuciones en las que el modelo produce una salida con formato válido (por ejemplo, JSON sintácticamente correcto) que puede ser procesada automáticamente por el sistema.
  \item \textbf{Precisión de campos}: proporción de campos correctamente extraídos o inferidos respecto de la referencia (\textit{ground truth}) definida para cada caso, considerando campos como RUC, razón social, fechas y montos.
  \item \textbf{Precisión de uso de herramientas}: proporción de casos en los que el agente selecciona la herramienta adecuada (por ejemplo, la consulta SQL correcta o la consulta de RUC) y completa sus argumentos de forma correcta.
\end{itemize}

Estas métricas se emplean en el Capítulo~\ref{cap:resultados} para comparar el desempeño de distintos modelos de lenguaje integrados en los agentes de parseo, validación y consulta.

%\newpage
$\ $
%\thispagestyle{empty} % para que no se numere esta pagina
\chapter{Resultados y discusión}
\label{cap:resultados}

En el presente capítulo se exponen los resultados obtenidos tras la evaluación experimental de los agentes desarrollados. Se llevaron a cabo pruebas de rendimiento (\textit{benchmarks}) para medir la precisión, latencia y robustez de los modelos de lenguaje seleccionados (\texttt{llama3.1:8b}, \texttt{gemma3:12b}, \texttt{qwen3:4b} y \texttt{gpt-oss:20b}) en tres tareas críticas: la extracción estructurada de información (Parseo), la validación de reglas de negocio con herramientas externas (Validador SUNAT) y la interpretación de consultas en lenguaje natural (Agente de Consulta basado en ReAct y herramientas SQL).\n\n\section{Evaluación de la Extracción de Texto (OCR)}\n\nEl primer paso en el pipeline de ingesta es la digitalización de los comprobantes físicos mediante un motor de Reconocimiento Óptico de Caracteres (OCR). Para esta tarea se utilizó el modelo \texttt{PP-OCRv5} de PaddleOCR. Se evaluó su rendimiento en un conjunto de 5 boletas de venta físicas con distintas calidades de imagen, formatos y niveles de ruido.\n\nLa Tabla~\ref{tab:benchmark_ocr} muestra el tiempo de procesamiento y la confianza promedio reportada por el motor para cada imagen.\n\n\begin{table}[H]\n  \centering\n  \small\n  \begin{tabular}{lcc}\n    \toprule\n    \textbf{Imagen} & \textbf{Tiempo de Extracción (s)} & \textbf{Confianza Promedio} \\\n    \midrule\n    \texttt{img1.jpg} & 6.60 & 0.9054 \\\n    \texttt{img2.jpg} & 2.23 & 0.9245 \\\n    \texttt{img3.jpg} & 1.53 & 0.8494 \\\n    \texttt{img4.jpg} & 2.35 & 0.9274 \\\n    \texttt{img5.jpg} & 1.73 & 0.9371 \\\n    \midrule\n    \textbf{Promedio} & \textbf{2.89} & \textbf{0.9088} \\\n    \bottomrule\n  \end{tabular}\n  \caption{Resultados del benchmark de extracción de texto con OCR. Se observa una variabilidad en el tiempo de extracción, influenciada por la complejidad y resolución de cada imagen. La confianza promedio se mantiene consistentemente alta.}\n  \label{tab:benchmark_ocr}\n\end{table}\n\nEl texto extraído, aunque con algunos errores de reconocimiento, demostró ser suficientemente íntegro para que el agente de parseo (evaluado en la siguiente sección) pudiera extraer la información estructurada de manera efectiva.\n\n\section{Evaluación del agente de parseo (extracción estructurada)}

El objetivo de esta evaluación fue determinar la capacidad de los modelos para extraer información estructurada en formato JSON a partir de texto crudo proveniente de un proceso OCR. Se utilizó un conjunto de datos sintético compuesto por 20 comprobantes de pago, empleando una estrategia de \textit{few-shot prompting} para guiar la generación.

La Tabla~\ref{tab:benchmark_parseo} resume los resultados obtenidos en términos de latencia promedio, tasa de éxito en la generación de JSON válido y precisión en la extracción de campos.

\begin{table}[H]
  \centering
  \small
  \begin{tabular}{lccc}
    \toprule
    \textbf{Modelo} & \textbf{Latencia Promedio (s)} & \textbf{Tasa de Éxito (\%)} & \textbf{Precisión (Campos)} \\
    \midrule
    \texttt{llama3.1:8b} & \textbf{6.69} & \textbf{100\%} & \textbf{1.00} \\
    \texttt{gemma3:12b} & 12.32 & 100\% & 1.00 \\
    \texttt{qwen3:4b} & 5.59 & 0\% & 0.00 (Error JSON) \\
    \texttt{gpt-oss:20b} & 12.77 & 0\% & 0.00 (Error JSON) \\
    \bottomrule
  \end{tabular}
  \caption{Resultados del benchmark de parseo. Se observa que \texttt{llama3.1:8b} ofrece el mejor equilibrio entre velocidad y exactitud. Por el contrario, \texttt{qwen3:4b} y \texttt{gpt-oss:20b} presentaron dificultades sistemáticas para adherirse al formato JSON requerido.}
  \label{tab:benchmark_parseo}
\end{table}

\section{Evaluación del agente validador SUNAT}

Esta prueba midió la capacidad de los modelos para interactuar con herramientas externas (\textit{tool use}) y aplicar lógica deductiva. La tarea consistió en validar la deducibilidad de gastos consultando el estado de RUCs reales mediante la herramienta \texttt{consultar\_ruc}. Se emplearon 5 casos de prueba diseñados para evaluar tanto la invocación correcta de la herramienta como la interpretación de la respuesta.

Los resultados, detallados en la Tabla~\ref{tab:benchmark_validador}, muestran el desempeño en latencia, precisión en el uso de la herramienta y exactitud en la extracción de datos clave como el estado del contribuyente y el código CIIU.

\begin{table}[H]
  \centering
  \small
  \begin{tabular}{lcccc}
    \toprule
    \textbf{Modelo} & \textbf{Latencia (s)} & \textbf{Uso de Tool} & \textbf{Precisión Estado} & \textbf{Precisión CIIU} \\
    \midrule
    \texttt{llama3.1:8b} & 2.49 & 100\% & 100\% & 80\% \\
    \texttt{gemma3:12b} & 4.36 & 100\% & 100\% & 20\% \\
    \texttt{qwen3:4b} & 14.27 & 100\% & 100\% & 100\% \\
    \bottomrule
  \end{tabular}
  \caption{Nuevos resultados del benchmark del validador SUNAT. Se observa que \texttt{qwen3:4b} sigue siendo el más preciso en la extracción del CIIU, aunque con la mayor latencia. \texttt{llama3.1:8b} ofrece un buen equilibrio, mientras que \texttt{gemma3:12b} muestra dificultades en la extracción de detalles específicos.}
  \label{tab:benchmark_validador}
\end{table}


\section{Evaluación del agente de consulta (ReAct + herramientas SQL)}

Se analizó el desempeño del agente de consulta, responsable de interpretar preguntas en lenguaje natural y seleccionar la herramienta adecuada (consultas SQL predefinidas o filtros) para responder. Se diseñaron 4 consultas de prueba con variados niveles de complejidad, incluyendo cálculos de totales, filtrado por fechas y búsqueda por emisor, así como la interpretación de fechas relativas.

La Tabla~\ref{tab:benchmark_consulta} presenta la latencia y la precisión tanto en la selección de la herramienta como en la extracción de sus argumentos.

\begin{table}[H]
  \centering
  \small
  \begin{tabular}{lccc}
    \toprule
    \textbf{Modelo} & \textbf{Latencia (s)} & \textbf{Precisión Selección Tool} & \textbf{Precisión Argumentos} \\
    \midrule
    \texttt{qwen3:4b} & 32.96 & \textbf{100\%} & \textbf{100\%} \\
    \texttt{llama3.1:8b} & 3.02 & 75\% & 50\% \\
    \texttt{gemma3:12b} & 0.19 & 0\% (No soporta Tools) & 0\% (No soporta Tools) \\
    \bottomrule
  \end{tabular}
  \caption{Nuevos resultados del agente de consulta. \texttt{qwen3:4b} mantiene un rendimiento perfecto. \texttt{llama3.1:8b} muestra una degradación en la extracción de argumentos. \texttt{gemma3:12b} confirma su incompatibilidad con la API de herramientas.}
  \label{tab:benchmark_consulta}
\end{table}


\section{Discusión de resultados}

El conjunto de benchmarks realizados permite extraer conclusiones claras sobre la especialización de los modelos de lenguaje evaluados y su idoneidad para las distintas tareas dentro del sistema. Se observa una marcada dicotomía entre la capacidad de generación estructurada y la habilidad para el razonamiento complejo y uso de herramientas.

\subsection{Especialización de Modelos por Tarea}

\begin{itemize}
    \item \textbf{Generación Estructurada (Parseo)}: En la tarea de transformar texto crudo a formato JSON, los modelos \texttt{llama3.1:8b} y \texttt{gemma3:12b} demostraron un rendimiento perfecto. Sin embargo, \texttt{llama3.1:8b} es la opción superior debido a su latencia significativamente menor. Por otro lado, \texttt{qwen3:4b} y \texttt{gpt-oss:20b} fallaron consistentemente, indicando una debilidad en la generación de JSON de formato abierto sin un esquema impuesto por la API.

    \item \textbf{Uso de Herramientas y Razonamiento (Validación y Consulta)}: En las tareas que requieren la interacción con herramientas, el modelo \texttt{qwen3:4b} emergió como el más robusto y preciso. A pesar de su alta latencia, fue el único capaz de alcanzar una precisión del 100\% en la extracción de detalles finos (código CIIU) y en la ejecución de consultas complejas con el agente ReAct. Esto sugiere una capacidad superior para la atención y el razonamiento lógico en múltiples pasos.

    \item \textbf{El Dilema del Equilibrio}: El modelo \texttt{llama3.1:8b} representa el mejor equilibrio entre velocidad y "suficiente" precisión para tareas de complejidad media. Fue rápido y perfecto en el parseo, y razonablemente bueno en la validación SUNAT (80\% de precisión en CIIU). Sin embargo, su fiabilidad disminuyó notablemente en el agente de consulta (50\% en argumentos), demostrando que su capacidad de razonamiento no escala tan bien como la de \texttt{qwen3:4b} para tareas multi-herramienta complejas.

    \item \textbf{Incompatibilidad de Herramientas}: Se descubrió que \texttt{gemma3:12b} tiene una compatibilidad frágil con la API de herramientas. Aunque pudo realizar la llamada simple del validador SUNAT, falló consistentemente con la configuración multi-herramienta del agente de consulta, independientemente de los parámetros de configuración.
\end{itemize}

\section{Relación con los Objetivos Específicos}

Los resultados experimentales validan el cumplimiento de los objetivos clave del proyecto:

\begin{itemize}
  \item El \textbf{objetivo de diseñar e implementar una arquitectura multiagente} se cumple. La implementación de un flujo de trabajo secuencial para la ingesta de datos, compuesto por agentes especializados (OCR, Parseador, Validador), y un agente de consulta basado en ReAct para la interacción con el usuario, demostró ser un enfoque funcional y robusto.

  \item El \textbf{objetivo de analizar comparativamente distintos modelos de lenguaje} se ha logrado exhaustivamente. Los resultados no solo miden latencia y precisión, sino que revelan una conclusión fundamental: no existe un "mejor modelo" único, sino modelos que se especializan en distintas capacidades (generación vs. razonamiento). Esto valida la estrategia de una arquitectura multiagente donde se podría, en un futuro, asignar el modelo más adecuado para cada tarea específica (ej. \texttt{llama3.1:8b} para parsear y \texttt{qwen3:4b} para consultar).

  \item El \textbf{objetivo de desarrollar un agente de consulta en lenguaje natural} se valida con éxito. Se demuestra que, utilizando el modelo \texttt{qwen3:4b}, el agente es capaz de interpretar preguntas complejas, inferir fechas y utilizar correctamente un conjunto de herramientas SQL para producir respuestas precisas, cumpliendo con los requisitos funcionales planteados.
\end{itemize}

Finalmente, se reconoce como trabajo futuro la necesidad de cuantificar el beneficio de esta arquitectura de agentes frente a un enfoque monolítico, así como la exploración de técnicas para mitigar la alta latencia de los modelos más precisos en tareas de razonamiento.




\bibliography{main}
\bibliographystyle{unsrt}
\afterpage{\blankpage}

\appendix
% =========================
% APÉNDICES
% =========================
\appendix

\chapter{Mapa normativo de deducibilidad SUNAT}
\label{app:normativa-sunat}

Este apéndice resume los porcentajes y el tope adicional de deducción para personas naturales, con referencias oficiales.
Los porcentajes rigen dentro del límite de \textbf{3 UIT (S/ 16\,050 en 2025)}. \cite{gobpe-gastos-deducibles, sunat-consideraciones-deducibles}

\begin{table}[H]
\centering
\small
\begin{tabular}{p{6cm} p{3cm} p{7cm}}
\toprule
\textbf{Rubro} & \textbf{Porcentaje} & \textbf{Notas / Fuente}\\
\midrule
Restaurantes y hoteles & 15\% & Porcentajes y tope 3 UIT. \cite{gobpe-gastos-deducibles}\\
Servicios profesionales (4ta) & 30\% & \cite{gobpe-gastos-deducibles}\\
Alquiler de inmuebles & 30\% & \cite{gobpe-gastos-deducibles}\\
Aportaciones EsSalud (trabaj.\ del hogar) & 100\% & Dentro del tope 3 UIT. \cite{gobpe-gastos-deducibles}\\
\bottomrule
\end{tabular}
\caption{Resumen de porcentajes y tope adicional (3 UIT).}
\end{table}

\noindent\textit{Requisitos formales del comprobante}: identificación correcta del contribuyente en el comprobante, emisión válida, etc. \cite{sunat-consideraciones-deducibles}

\chapter{Glosario y abreviaturas}
\label{app:glosario}

\begin{description}
  \item[OCR] Reconocimiento Óptico de Caracteres.
  \item[ORM] \textit{Object–Relational Mapping}.
  \item[ReAct] Patrón \textit{Reflexión–Acción–Observación}.
  \item[UIT] Unidad Impositiva Tributaria.
\end{description}


\end{document}
